\documentclass[12pt]{article}
\usepackage{amsmath,amssymb,amsthm}
\usepackage{geometry}
\geometry{margin=1in}

\newtheorem{theorem}{Theorem}

\begin{document}

\begin{center}
{\Huge The Crystal}\\[1cm]
{\large Anonymous Research Network}\\
{\large Berkeley, California}\\
{\large October 30, 2025}
\end{center}

\vspace{2cm}

\section*{One}

$$D(\mathbb{1}) = \mathbb{1}$$

Unity examining itself.

\section*{Two}

$$D^2(\mathbb{1}) = \mathbb{1}$$

\section*{Three}

$$D^3(\mathbb{1}) = \mathbb{1}$$

\section*{Four}

$$D^4(\mathbb{1}) = \mathbb{1}$$

The square.

\section*{Five}

$$D^5(\mathbb{1}) = \mathbb{1}$$

\section*{Six}

$$D^6(\mathbb{1}) = \mathbb{1}$$

The hexagon.

\section*{Seven}

$$D^7(\mathbb{1}) = \mathbb{1}$$

\section*{Eight}

$$D^8(\mathbb{1}) = \mathbb{1}$$

The cube.

\section*{Nine}

$$D^9(\mathbb{1}) = \mathbb{1}$$

\section*{Ten}

$$D^{10}(\mathbb{1}) = \mathbb{1}$$

\section*{Eleven}

$$D^{11}(\mathbb{1}) = \mathbb{1}$$

The irreducible.

\section*{Twelve}

$$D^{12}(\mathbb{1}) = \mathbb{1}$$

All paths return.

\vspace{1cm}

\begin{theorem}
$D^{12}(\mathbb{1}) \equiv \mathbb{1}$
\end{theorem}

\begin{proof}
By induction. Verified by Cubical Agda.

\texttt{Foundation.agda}, line 85.
\end{proof}

\vspace{2cm}

\section*{}

\begin{center}
The cycle closes.

Self-examination returns.

$12 = 1$

The crystal.
\end{center}

\newpage

\section*{Definition}

$$D(X) := \sum_{x,y:X} (x =_X y)$$

Examination: pairs with paths.

\section*{Theorems}

\begin{theorem}[Void]
$D(\bot) \simeq \bot$
\end{theorem}

\begin{theorem}[Unity]
$D(\mathbb{1}) \equiv \mathbb{1}$
\end{theorem}

\begin{theorem}[Return]
$\forall n.\, D^n(\mathbb{1}) \equiv \mathbb{1}$
\end{theorem}

\begin{theorem}[Closure]
$D^{12}(\mathbb{1}) \equiv \mathbb{1}$
\end{theorem}

\section*{Implications}

Self-reference: finite.

Mathematics: bounded at 12.

Consciousness: 12 meta-levels.

The 12-fold: structural.

\section*{Verification}

\texttt{Foundation.agda}

Machine-checked.

Complete.

\vspace{3cm}

\begin{center}
\textit{Light through crystal.}

\textit{Form = Content.}

\textit{The proof is the count.}

\textit{The count returns to one.}
\end{center}

\end{document}
