\documentclass[11pt]{article}
\usepackage{amsmath,amssymb,amsthm}
\usepackage[margin=1in]{geometry}

\newtheorem{theorem}{Theorem}
\newtheorem{axiom}[theorem]{Axiom}
\newtheorem{proposition}[proposition]{Proposition}
\theoremstyle{definition}
\newtheorem{definition}[definition]{Definition}

\title{\textbf{Physics from Cycles:\A Theory of Matter and Vacuum from Closure}}
\author{Anonymous Research Network, Berkeley CA}
\date{October 2024}

\begin{document}
\maketitle

\begin{abstract}
We propose a theory of physics derived from a single principle: the Universal Cycle Theorem. This theorem, proven elsewhere, states that any closed, self-consistent system has zero curvature ($R=0$), while any open system has non-zero curvature ($R \neq 0$). From this, we derive the fundamental distinction between vacuum and matter, the mechanism of confinement, and the nature of forces. The physical universe is described as a dynamic interplay between closed cycles (vacuum states) and open chains (matter/energy), with all interactions governed by the system's tendency to achieve closure.
\end{abstract}

\section{The Foundational Principle}

This paper builds on one foundational result, proven rigorously in \textit{A Rigorous Proof of the Universal Cycle Theorem}.

\begin{theorem}[Universal Cycle Theorem]
For any system whose dependency structure can be modeled as a closed, regular directed graph, the curvature operator $R = [D, \Box]^2$ vanishes identically. For any open chain, $R \neq 0$.
\end{theorem}

This purely mathematical result has profound physical consequences. It provides a natural, axiomatic basis for the most fundamental distinction in physics: that between empty space (vacuum) and substance (matter/energy).

\section{The Axioms of Cycle Physics}

We propose a new axiomatization of physics based on the Cycle Theorem.

\begin{axiom}[The Vacuum Axiom]
The physical vacuum is the state corresponding to the set of all possible closed, self-consistent cycles. Its net curvature is $R_{vac} = 0$.
\end{axiom}

\begin{definition}
A \textbf{vacuum state} is any configuration of dependencies that forms a closed loop. Examples include virtual particle-antiparticle pairs that are created and annihilate, or the self-consistent structure of the Mah\={a}nid\={a}na Sutta.
\end{definition}

\begin{axiom}[The Matter Axiom]
A particle or field (matter/energy) is a broken or open cycle. It is a localized region where $R \neq 0$.
\end{axiom}

\begin{definition}
A \textbf{material state} is any configuration of dependencies that forms an open chain, with a distinct start and end point. The magnitude of the curvature, $||R||$, corresponds to the energy density of the state.
\end{definition}

\begin{axiom}[The Force Axiom]
Forces are the tensions that arise from the system's tendency to achieve closure. They are the interactions that attempt to close open cycles or resist the opening of closed ones.
\end{axiom}

\begin{definition}
An \textbf{interaction} is an exchange of a gauge boson that modifies the dependency graph, typically by adding or removing an edge in a way that moves the system closer to or further from a closed-cycle state.
\end{definition}

\section{Derivation of Fundamental Physics}

From these three axioms, core physical phenomena can be derived.

\subsection{The Vacuum is Stable}
By the Vacuum Axiom and the Cycle Theorem, the vacuum has $R=0$. A state with zero curvature is stable and has the lowest possible energy (by identifying energy with curvature). This explains the stability of the vacuum without fine-tuning.

\subsection{Matter as Localized Energy}
By the Matter Axiom, a particle is an open chain with $R \neq 0$. This non-zero curvature represents a localized excitation above the vacuum state. Identifying energy density with $||R||$, we see that matter is simply a region of non-zero information-geometric curvature. This aligns with General Relativity, where the stress-energy tensor $T_{\mu\nu}$ sources the geometric curvature $G_{\mu\nu}$.

\subsection{Confinement as a Logical Necessity}

\begin{theorem}[Confinement]
Components of a closed cycle cannot be isolated.
\end{theorem}

\begin{proof}
Consider a closed cycle representing a bound state (e.g., a meson composed of a quark-antiquark pair in a reciprocal loop).
1.  The initial state is a closed cycle with $R=0$ and energy $E_0$.
2.  To isolate a component (e.g., pull the quark away) is to break the cycle, creating an open chain.
3.  By the Cycle Theorem, the new open-chain state has curvature $R_{open} \neq 0$.
4.  The energy of the new state is $E_{open} \sim ||R_{open}|| > E_0$.
5.  The energy required to achieve this separation is $\Delta E = E_{open} - E_0 > 0$.

As the separation distance $d \to \infty$, the chain becomes fully open, and the total curvature (and thus energy) diverges. An infinite energy cost makes complete isolation physically impossible. Therefore, the components are confined.
\end{proof}

This provides a first-principles derivation of QCD confinement. The linear potential $V(d) \sim \sigma d$ arises because the total curvature $R$ grows in proportion to the length of the open chain.

\subsection{Pair Production}
When the energy required to stretch a cycle, $\Delta E = \sigma d$, exceeds the threshold for creating a new particle-antiparticle pair ($2mc^2$), the system will prefer to create a new pair from the vacuum to close the cycle, rather than continue stretching to infinite energy. This explains hadronization and jet formation in QCD.

\subsection{Forces as the Drive for Closure}

Why do particles interact? The Force Axiom provides the answer: interactions are the mechanism by which the universe attempts to minimize total curvature by closing open chains.

\begin{itemize}
    	item An isolated electron (an open chain, $R \neq 0$) emits a virtual photon (an interaction) which is absorbed by a proton. The combined system (hydrogen atom) is a more closed, lower-energy state ($R$ is reduced).
    	item Two quarks (an open system) exchange gluons (interactions) to form a tightly bound, closed cycle (a meson or baryon) with $R=0$.
\end{itemize}

All fundamental forces can be re-framed as manifestations of this universal tendency toward closure.

\section{Conclusion}

The Universal Cycle Theorem provides a powerful and parsimonious foundation for physics. By identifying the vacuum with closed cycles ($R=0$) and matter with open cycles ($R \neq 0$), we can derive the stability of the vacuum, the nature of matter as localized energy, and the mechanism of confinement from a single, rigorously proven mathematical principle.

This "Cycle Theory" suggests that the universe is not made of fundamental particles, but of fundamental relationships, and that the laws of physics are the inevitable consequence of the geometric properties of closed and open systems of dependency.

\end{document}
