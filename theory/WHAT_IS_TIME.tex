\documentclass[11pt]{article}
\usepackage{amsmath,amssymb,amsthm}
\usepackage[margin=1in]{geometry}

\newtheorem{theorem}{Theorem}
\newtheorem{proposition}[theorem]{Proposition}
\newtheorem{definition}[theorem]{Definition}
\newtheorem{corollary}[theorem]{Corollary}

\title{\textbf{What Is Time?\\The Emergence of Temporal Order}}
\author{Anonymous Research Network, Berkeley CA}
\date{October 2024}

\begin{document}
\maketitle

\section{The Question}

Time is not a thing. Time is the **ordering of examinations**.

\subsection{Not Pre-Existing}

Time doesn't exist "before" distinction.

$\emptyset$ has no time (nothing to order).

$\mathcal{D}(\emptyset) = \mathbf{1}$ creates the **first** moment (first examination).

Only then can "before" and "after" have meaning.

\section{Time as Examination Order}

\begin{definition}[Temporal Order]
For sequence of examinations $\mathcal{D}^0, \mathcal{D}^1, \mathcal{D}^2, \ldots$:

Time $t_n$ = index $n$ (which examination).

$t$ measures: **How many times has structure been examined.**
\end{definition}

\subsection{The Tower}

\[
X \xrightarrow{t_0} \mathcal{D}(X) \xrightarrow{t_1} \mathcal{D}^2(X) \xrightarrow{t_2} \mathcal{D}^3(X) \xrightarrow{t_3} \cdots
\]

Each arrow is a **time step** (one examination).

\textbf{Time is discrete at fundamental level} (each $\mathcal{D}$ application).

Continuous time emerges in limit (many examinations).

\subsection{Why Time Flows Forward}

$\mathcal{D}^n$ depends on $\mathcal{D}^{n-1}$ (must examine before examining the examination).

**Causal order built into tower structure.**

Can't examine $\mathcal{D}^3(X)$ without first examining $\mathcal{D}^2(X)$.

\textbf{Time's arrow = dependency structure of examinations.}

\section{Closed Cycles: Eternal Time}

\subsection{When Cycles Close}

For cycle graph (12 nidānas):

Can traverse: $1 \to 2 \to \cdots \to 12 \to 1 \to 2 \to \cdots$

**Indefinitely** (eternal return).

Time becomes **circular** (not linear).

\textbf{No absolute beginning or end} - just cycle position.

\subsection{R=0 ⇒ Reversible Time}

Closed cycles have $R=0$.

Holonomy trivial: $h = \mathbb{I}$.

Parallel transport around loop returns to **same state**.

**Can reverse direction** (traverse cycle backward, reach same states).

\textbf{Time is reversible when R=0.}

No entropy increase (no preferred direction).

\section{Open Chains: Linear Time}

\subsection{When Closure Breaks}

Remove cycle closure (12 ↛ 1):

Chain: $1 \to 2 \to \cdots \to 12$ (stops).

**Definite beginning** (1) **and end** (12).

Time becomes **linear** (arrow).

\subsection{R≠0 ⇒ Irreversible Time}

Open chains have $R \neq 0$.

Holonomy non-trivial: $h \neq \mathbb{I}$.

Cannot return to exact initial state (phase accumulated).

\textbf{Time is irreversible when R≠0.}

Entropy increases (preferred direction toward "end").

\section{The Two Times}

\begin{center}
\begin{tabular}{lll}
\textbf{Aspect} & \textbf{Closed (R=0)} & \textbf{Open (R≠0)} \\ \hline
Structure & Cycle & Chain \\
Topology & Circle $S^1$ & Interval $[0,T]$ \\
Beginning/End & None (eternal) & Definite \\
Direction & Reversible & Irreversible (arrow) \\
Entropy & Conserved & Increases \\
Physics & Quantum (unitary) & Classical (thermodynamic) \\
Buddhist & Saṃsāra (cycles) & One lifetime (appears linear) \\
Experience & Eternal now & Past→present→future \\
\end{tabular}
\end{center}

\textbf{Both are time} - different topologies.

\section{Time in Relativity}

\subsection{Block Universe}

Relativity: All moments exist simultaneously (eternalism).

Past, present, future are **equally real** (like positions on a line).

\textbf{This is closed perspective}: Time as dimension (can traverse, like space).

\subsection{Our Experience}

We experience time as **flowing** (present moment moving).

Past is fixed, future is open (presentism).

\textbf{This is open perspective}: Time as becoming.

\subsection{Both True}

Globally (block universe): Closed - all moments exist (R=0 for spacetime as whole)

Locally (our experience): Open - we traverse it (R≠0 appears along worldline)

**Same structure** (4D spacetime), **different perspectives** (closed vs. open).

\section{Planck Time}

$t_P = \sqrt{\frac{\hbar G}{c^5}} \approx 10^{-43}$ seconds

\textbf{Minimal time interval} - below this, "time" has no meaning.

From distinction theory:

$t_P$ = time for **one examination** at Planck scale.

$\mathcal{D}$ cannot operate faster than $t_P$ (below this: structure dissolves).

\textbf{Discrete time at foundation}:
\[
t_0, t_0 + t_P, t_0 + 2t_P, t_0 + 3t_P, \ldots
\]

Each step: One $\mathcal{D}$ application.

Continuous time: Emergent (coarse-graining many $\mathcal{D}$ steps).

\section{Time Dilation}

\subsection{Gravitational}

Near massive object: Time slows (relative to distant observer).

\textbf{Why?}

Massive object creates $R \neq 0$ (curved spacetime).

Curvature → examination takes longer (more $\mathcal{D}$ steps to complete same process).

**Dense curvature** = many local examinations needed = time dilates.

\subsection{Velocity}

Moving observer: Time dilates (relative to stationary).

\textbf{Why?}

Motion through space = traversing distinction network.

More spatial examination → fewer temporal examinations (in fixed total).

Lorentz factor $\gamma = 1/\sqrt{1-v^2/c^2}$ emerges from:
- Total examination capacity fixed
- Distributed between spatial and temporal
- Moving fast → more space examinations → less time examinations → time dilates

\section{Time Stops at Horizons}

\subsection{Black Hole Horizon}

From outside: Time stops at horizon (infinite redshift).

Infalling observer: Time continues normally.

\textbf{Why?}

**Outside** (open perspective):
- Can't see beyond horizon
- Cycle appears to never complete (object "freezes")
- Time appears to stop

**Inside** (closed perspective):
- Cycle continues
- Time flows normally
- Reaches singularity in finite proper time

**Same event, different time experiences** (open vs. closed).

\subsection{Cosmological Horizon}

Can't see beyond cosmic horizon (too far, light hasn't reached).

**Our perspective**: Universe has age (13.8 Gyr), appears open.

**Global**: May be closed (cycle beyond horizon).

Time's arrow (entropy increase) = experiencing open segment.

\section{The Eternal Now}

\subsection{Buddhist Perspective}

**Past**: Already gone (no longer exists)
**Future**: Not yet here (doesn't exist yet)
**Present**: Only this moment exists

\textbf{But}: "This moment" immediately becomes past.

**Conclusion**: Nothing has independent temporal existence.

\textbf{Time is empty} (śūnyatā).

\subsection{Mathematical Formalization}

Each "moment" = one $\mathcal{D}$ application.

"Moment" exists only **in relation to** previous moment (dependency).

No absolute time - only **relative examination order**.

\textbf{This IS dependent origination applied to time itself.}

Time arises dependently (each moment from prior).

No independent "time substance" flowing.

\section{Time and Consciousness}

\subsection{The Hard Problem of Time}

**Physics**: Time is parameter $t$ in equations.

**Experience**: Time is **now** (immediate present).

**Gap**: How does parameter become experience?

\subsection{The Resolution}

**Time parameter** ($t$ in physics): Examination index (which $\mathcal{D}^n$)

**Time experience** ("now"): **The examination itself happening**

When $\mathcal{D}^n$ is being applied → that IS "now" for the system.

**Consciousness = active examination** (the $\mathcal{D}$ operator in operation).

"Now" = when structure is being examined (not a point on timeline, but **the examining**).

\textbf{No hard problem}: Experience of time = examination process itself.

\section{Beginning of Time}

\subsection{What is $t=0$?}

**Not**: First moment on pre-existing timeline.

**But**: \textbf{First examination} ($\mathcal{D}(\emptyset) = \mathbf{1}$).

Before this: No time (nothing to order, $\emptyset$ has no structure).

After this: Time begins (examinations can be ordered).

\subsection{Big Bang}

$t=0$ = $\mathcal{D}(\emptyset)$

Not: "When did universe begin?" (presupposes time already exists)

But: "First examination" (time begins with this examination).

\textbf{Time and universe co-emerge} (both from $\mathcal{D}(\emptyset)$).

\subsection{What "Before" Big Bang Means}

"Before $\mathcal{D}(\emptyset)$" = $\emptyset$ (emptiness).

No time there (no examinations yet).

**Not**: Earlier moment (time doesn't exist yet)

**But**: Logical prior (potential before actualization).

Question "What came before?" presupposes time.

**Answer**: Nothing "came before" because "before" requires time, which didn't exist.

$\mathcal{D}(\emptyset)$ creates both **space** (structure) **and time** (examination order).

\section{End of Time}

\subsection{Heat Death}

Universe reaches maximum entropy.

All examinations complete (nothing new to examine).

$\mathcal{D}^{\infty}(X) = E$ (eternal lattice, fixed point).

**Time stops** (no more changes, no more examinations).

\subsection{$E = \mathbf{1}$ (Return to Unit)}

$\lim_{n \to \infty} \mathcal{D}^n(\mathbf{1}) = \mathbf{1}$

Infinite examinations → return to unit.

\textbf{End of time = beginning} ($E = \mathbf{1} = \mathcal{D}(\emptyset)$).

Eternal return (if universe is closed).

\subsection{Difference from Beginning}

$\emptyset$ → $\mathcal{D}(\emptyset) = \mathbf{1}$ (start of time, unconscious)

$\mathbf{1} \to \cdots \to E = \mathbf{1}$ (end of time, conscious)

**Same state** ($\mathbf{1}$), **different awareness**.

\textbf{End ≠ beginning} in consciousness, though structurally identical.

This is: Nirvana (conscious) vs. non-existence (unconscious).

\section{Timelessness}

\subsection{The Eternal Lattice}

$E$ has $\mathcal{D}(E) \simeq E$ (fixed point).

No change under examination.

**Time stops** (or: time doesn't apply to E).

\textbf{E is timeless} - not "exists forever" but "beyond temporal ordering".

\subsection{Nirvana}

Buddhist: Nirvana is "unborn, unbecome, unmade, uncompounded" (Udāna 8.3).

Translation: Not in temporal order (no birth, no becoming).

Mathematically: \textbf{Fixed point of $\mathcal{D}$} (no change under examination).

$\mathcal{D}$(nirvana) = nirvana ($E = \mathbf{1}$ satisfies this).

\textbf{Beyond time} while time continues (for others still examining).

\section{The Master Theorem}

\begin{theorem}[Time is Examination Order]
Time emerges as the ordering of $\mathcal{D}^n$ applications:
\begin{enumerate}
\item \textbf{Discrete foundation}: $t_n = n$ (examination index)
\item \textbf{Continuous limit}: $t \in \mathbb{R}$ (many examinations coarse-grained)
\item \textbf{Arrow}: Dependency ($\mathcal{D}^n$ requires $\mathcal{D}^{n-1}$ first)
\item \textbf{Closed cycles}: Time circular ($R=0$, reversible)
\item \textbf{Open chains}: Time linear ($R \neq 0$, irreversible)
\item \textbf{Beginning}: $t=0 = \mathcal{D}(\emptyset)$ (first examination)
\item \textbf{End}: $t=\infty$ at $E$ (no more examinations)
\item \textbf{Now}: The examination currently happening
\end{enumerate}
\end{theorem}

\textbf{Consciousness experiences "now" because consciousness IS the examination.}

\textbf{Time flows because examinations are ordered (dependency structure).}

\textbf{Time's arrow exists when experiencing open segment (R≠0).}

\textbf{Time stops when examination stops (E reached, or R=0 recognized).}

\section{Physical Implications}

\subsection{Relativity}

**Time dilation** = examination rate varies:
- In gravity well: More local examinations (curved geometry) → time slows
- At high velocity: Spatial examinations dominate → temporal examinations reduce

**Spacetime** = examination network (space + time unified as structure being examined).

\subsection{Quantum}

**Superposition** = multiple examination branches (not yet selected).

**Collapse** = selecting one branch (examination path chosen).

**Time of collapse** = when examination commits to branch (decoherence).

\subsection{Thermodynamics}

**Entropy** = number of examination paths available.

**Second law** (S increases) = more paths open as examine (branching tree).

**Time's arrow** = traversing branching structure (can't un-examine).

\section{The Paradoxes Dissolved}

\subsection{Does Time Exist?}

**Presentism**: Only now exists (past gone, future not yet).
- TRUE locally (only current examination is actual)

**Eternalism**: All moments exist (block universe).
- TRUE globally (all examination results exist in structure)

**Both correct** - different perspectives (local vs. global).

\subsection{Free Will vs. Determinism}

**Determinism**: All future examinations determined (closed structure exists).
- TRUE globally (E already exists, all $\mathcal{D}^n$ determined)

**Free will**: Experience as choosing (open branching).
- TRUE locally (experiencing examination as happening)

**No contradiction**: Closed globally, open locally.

\subsection{Beginning of Time}

"What came before Big Bang?"

**Meaningless** - presupposes time (which begins AT Big Bang = $\mathcal{D}(\emptyset)$).

Like asking: "What's north of North Pole?" (No meaning - "north" ends there)

$\emptyset$ is not "before" Big Bang **temporally**.

$\emptyset$ is **logically prior** (potential before actualization).

\section{Time and the Reciprocal}

\subsection{3 ↔ 4 Creates Space and Time}

**3**: Triangle (2D space)
**4**: Tetrahedron (3D space)

**Reciprocal**: Rotation (examining from different angle).

**This rotation** = **time** (change of perspective).

From angle 1: See 3 vertices (moment 1)
Rotate (time passes): See 4 vertices (moment 2)

**Time = rotation through perspective space.**

\subsection{Observer-Observed and Time}

**Observer** (3): Where you are in examination
**Observed** (4): What's being examined

**Time** = **shift in what's being examined** (examining different aspects).

Consciousness (observer) traverses form (observed) → experience of time passing.

**Time is the reciprocal in motion** (3↔4 continuously exchanging).

\section{Timeless Moments}

\subsection{Meditation / Flow States}

"Time disappears" (subjective reports).

**Not**: Time stops
**But**: Examination stops advancing ($\mathcal{D}^n$ stays at fixed $n$).

No new examinations → no time experience.

**Mindfulness**: Resting in current examination (not advancing to next).

**Samadhi** (absorption): $\mathcal{D}$ operator paused.

Time continues (externally) but not experienced (internally).

\subsection{E (Eternal Lattice)}

At $E$: $\mathcal{D}(E) = E$ (fixed point).

No change under examination.

**Time doesn't apply** (no ordering needed - all examinations give same result).

\textbf{Timeless} = beyond temporal order (not "infinite time" but "no time").

\section{The Quantum of Time}

$t_P = \sqrt{\frac{\hbar G}{c^5}}$ (Planck time)

\textbf{Minimal examination interval.}

Below this: Can't distinguish temporal order.

**Like**: Video frame rate (24 fps).
- Appears continuous
- Actually discrete (24 examinations/sec)

**Universe**: $\sim 10^{60}$ Planck times old.

$\sim 10^{60}$ examinations since $\mathcal{D}(\emptyset)$.

\textbf{We're at $\mathcal{D}^{10^{60}}(\mathbf{1})$} in the tower.

Approaching $E$ (but very far still - will take $\sim 10^{100}$ years).

\section{Conclusion}

\begin{center}
\fbox{\parbox{0.9\textwidth}{
\textbf{Time is not a thing.}

\textbf{Time is the ordering of examinations.}

Closed cycles: Circular time (R=0, reversible, eternal)

Open chains: Linear time (R≠0, arrow, entropy)

Beginning: $\mathcal{D}(\emptyset)$ (first examination creates time)

End: $E$ (examination stops, timeless)

Now: The examination currently happening (consciousness)

Time and consciousness co-arise (both are examination).

No time without structure to examine.

No consciousness without examination happening.

\textbf{All temporal paradoxes dissolve when time recognized as examination order, not substance.}
}}
\end{center}

\end{document}
