\documentclass[11pt]{article}
\usepackage{amsmath,amssymb,amsthm}
\usepackage[margin=1in]{geometry}

\newtheorem{theorem}{Theorem}[section]
\newtheorem{proposition}[theorem]{Proposition}
\newtheorem{lemma}[theorem]{Lemma}
\newtheorem{corollary}[theorem]{Corollary}
\theoremstyle{definition}
\newtheorem{definition}[theorem]{Definition}
\theoremstyle{remark}
\newtheorem{remark}[theorem]{Remark}

\title{\textbf{Born Rule from Self-Examination:\\Probability as Intrinsic Closure Property}}
\author{Anonymous Research Network, Berkeley CA}
\date{October 2024}

\begin{document}
\maketitle

\begin{abstract}
We derive the Born rule by showing probability emerges from quantum state's \emph{self-examination}, not external observation. The key: $|\psi|^2$ is \textbf{one application of self-reference} (Δ=1) - the system examining itself creates stable intrinsic property. This property is gauge-invariant (independent of external observation angle), therefore becomes externally observable despite originating from internal self-examination. The squaring (quadratic) reflects that probability measures how state relates to \emph{itself}, not to external observer. This connects Born rule to Closure Principle: one iteration of self-examination suffices to create observable intrinsic property. Probability is not imposed by measurement but emerges from system's self-closure.
\end{abstract}

\section{The Central Question}

\subsection{Whose Observation Creates Probability?}

Standard quantum mechanics:

State $|\psi\rangle = \sum_i c_i |i\rangle$ (superposition)

Measurement by **external observer** gives outcome $i$ with probability $P(i) = |c_i|^2$.

\textbf{Problem}: Makes probability extrinsic (depends on observer existing).

\textbf{Question}: Before any external observer, does probability exist?

\textbf{If no}: Probability is human/observer construct (anti-realist)

\textbf{If yes}: Where does it come from? (realist)

\subsection{Our Answer}

\begin{center}
\fbox{\parbox{0.9\textwidth}{
\textbf{Probability emerges from system's self-examination.}

The quantum state examines \emph{itself} (applies $\mathcal{D}$ to self).

This creates intrinsic property $|\psi|^2$ (one self-reference iteration).

Property is stable (gauge-invariant, observer-independent).

Therefore: Externally observable (all observers see same value).

But originates: From internal self-examination (intrinsic, not imposed).
}}
\end{center}

\section{Self-Examination Creates Probability}

\subsection{The Mechanism}

\begin{definition}[Self-Examination of Quantum State]
State $|\psi\rangle$ examines itself via:
\[
\mathcal{D}_{\text{self}}(|\psi\rangle) = |\psi\rangle \otimes |\psi\rangle^* = |\psi\rangle\langle\psi|
\]

This is: State paired with its conjugate (examining self, not other).
\end{definition}

\textbf{Physical meaning}:

System doesn't need external observer to "know about itself."

**Internal dynamics** cause state to self-reference:
\begin{itemize}
\item Interactions with itself (self-energy)
\item Evolution under Hamiltonian (self-examination over time)
\item Quantum fluctuations (virtual processes = self-interaction)
\end{itemize}

**All are**: System examining its own structure.

\subsection{Squaring = One Self-Reference}

\begin{theorem}[Squaring is Δ=1]
The operation $|\psi| \to |\psi|^2$ is \textbf{one application of self-reference}:

\[
|\psi|^2 = \psi^* \cdot \psi = \langle\psi|\psi\rangle
\]

This is: **State with itself** (self-pairing).

Not: State with external reference.

\textbf{Squaring = single self-examination iteration (Δ=1).}
\end{theorem}

\begin{proof}
Linear: $|\psi|$ (state itself, no self-reference)

Quadratic: $|\psi|^2 = \psi \cdot \psi$ (state paired with self, one self-reference)

Cubic: $|\psi|^3 = \psi \cdot \psi \cdot \psi$ (two self-references: $\psi$ with $(\psi \cdot \psi)$)

**Born rule is quadratic because**: \textbf{One self-reference suffices} (Closure Principle).

More iterations (cubic, quartic) are redundant - property already stable after Δ=1.
\end{proof}

\section{Intrinsic Property Through Closure}

\subsection{How Internal Becomes Observable}

\begin{theorem}[Self-Closure Creates Observable]\label{thm:closure-observable}
When system self-examines (Δ=1 iteration):

Creates property that is:
\begin{enumerate}
\item \textbf{Intrinsic}: Originates from internal self-reference
\item \textbf{Stable}: Independent of external observation angle (gauge-invariant)
\item \textbf{Observable}: Because stable, all external observers see same value
\end{enumerate}
\end{theorem}

\begin{proof}
**Step 1 - Self-examination**:

System examines itself: $|\psi\rangle$ paired with $|\psi\rangle$ (internal operation, no external reference needed).

Result: $|\psi\rangle\langle\psi|$ (density operator).

**Step 2 - Gauge invariance**:

Under phase rotation $|\psi\rangle \to e^{i\alpha}|\psi\rangle$ (external gauge transformation):
\[
|\psi\rangle\langle\psi| \to e^{i\alpha}|\psi\rangle\langle\psi|e^{-i\alpha} = |\psi\rangle\langle\psi|
\]

**Invariant** (phase cancels in self-pairing).

**Step 3 - Stability achieved**:

Property $P(i) = \langle i|\psi\rangle\langle\psi|i\rangle = |c_i|^2$ is:
\begin{itemize}
\item Independent of global phase (gauge-invariant)
\item Independent of basis choice (unitary transformations preserve)
\item Independent of external observer (intrinsic)
\end{itemize}

**Step 4 - External observability**:

\textbf{Because} property is stable (invariant under all external transformations):

All external observers measure **same value** (objective).

**Property becomes observable** not because observer creates it, but because **it achieved closure** (gauge-invariant through self-pairing).

**Intrinsic property** (from self-examination) that's **externally detectable** (from closure/stability).
\end{proof}

\subsection{The Profound Implication}

**Hidden internal state** (complex amplitude $c_i$):
- Phase-dependent
- Not directly observable (hidden variable)
- Gauge-dependent (changes under $|\psi\rangle \to e^{i\alpha}|\psi\rangle$)

**After self-examination** (one Δ=1 iteration):
- Probability $|c_i|^2$ (real number)
- **Observable** (all observers agree)
- Gauge-invariant (phase self-cancels)

**The self-examination transforms**:
\begin{itemize}
\item Hidden → Visible
\item Phase-dependent → Phase-independent
\item Internal → Externally detectable
\end{itemize}

\textbf{This happens through closure} (Δ=1 achieves stability).

\textbf{Not magic} - mathematical necessity from self-pairing creating gauge-invariant combination.

\section{Why External Observers See Intrinsic Property}

\subsection{The Apparent Paradox}

\textbf{Seems contradictory}:
\begin{itemize}
\item Probability is intrinsic (from self-examination)
\item Yet: External observers detect it (seems extrinsic)
\end{itemize}

\textbf{Resolution}: Properties that achieve **closure through self-examination** become **objective** (observer-independent).

\subsection{The Mechanism}

\begin{proposition}[Closed Properties Are Objective]
If property $P$ emerges from self-examination (system examining itself):

And: $P$ is gauge-invariant (same under all external transformations)

Then: \textbf{All external observers see same $P$} (objective property).

Even though: $P$ originated internally (no external observer needed for creation).
\end{proposition}

\begin{proof}
**Gauge invariance** means: $P$ unchanged under U (unitary/symmetry transformation).

Different observers are related by transformations: $|\psi\rangle_A = U|\psi\rangle_B$

If $P$ gauge-invariant: $P_A = P_B$ (all observers agree).

Therefore: Property is **objective** (observer-independent).

**But**: Property was created by **self-examination** (internal), not by external measurement.

**External observers detect**: Pre-existing objective property.

**They don't create it**: System created it (via self-examination).
\end{proof}

\textbf{Example}:

Mass of electron: $m_e = 0.511$ MeV

This is **intrinsic** (electron has this mass regardless of observation)

Yet: **Observable** (all observers measure same value)

**Because**: Mass achieved through self-interaction (electron with itself via Higgs)

**Self-closure** → intrinsic property → gauge-invariant → observable by all.

**Same mechanism for probability.**

\section{The Depth-1 Structure}

\subsection{Why One Self-Reference Suffices}

**Closure Principle**: Δ=1 (one iteration of self-examination) creates stable property.

**Applied to Born rule**:

**Δ=0** (no self-reference): Amplitude $c_i$ (complex, hidden, gauge-dependent)

**Δ=1** (one self-reference): Probability $|c_i|^2 = c_i^* c_i$ (real, observable, gauge-invariant)

**One pairing** (state with self) creates:
- Phase cancellation ($e^{i\theta} \cdot e^{-i\theta} = 1$)
- Gauge invariance (independent of external angle)
- Stability (all observers agree)

**After this**: Property is fixed.

**More iterations** (Δ=2, Δ=3, ...): Redundant.

$|\psi|^2$ already stable - squaring again ($|\psi|^4$) doesn't add information about intrinsic state.

\textbf{Quadratic is necessary and sufficient} (Δ=1 closure).

\subsection{Universal Pattern}

Across all domains:

\begin{center}
\begin{tabular}{lll}
\textbf{System} & \textbf{Self-Reference Once} & \textbf{Result} \\ \hline
Quantum state & $|\psi\rangle$ with self & Probability $|\psi|^2$ \\
Number & $n$ with self & Square $n^2$ \\
Curvature & $\nabla$ with self & $R = \nabla^2$ \\
Logic & System examining consistency & Gödel (depth-1) \\
Spacetime & Point with self & Proper time $ds^2$ \\
\end{tabular}
\end{center}

**All quadratic from**: One self-reference iteration (Δ=1).

**This is not coincidence** - it's Closure Principle applied universally.

\section{Measurement as Recognition}

\subsection{What Does External Observer Do?}

If system already self-examined → probability already exists intrinsically...

**What is external measurement?**

\begin{definition}[Measurement = Recognition of Intrinsic Property]
External observer **detects** (not creates) the pre-existing probability:

System self-examined → $|c_i|^2$ exists intrinsically

Observer measures → **recognizes** this property (reads what's already there)
\end{definition}

**Analogy**:

Tree falls in forest (no human observers).

Tree examines itself (via gravity, structure, forces) → has intrinsic property (fallen-ness).

Human arrives later → **detects** this property (tree is fallen).

**Human didn't create** the property (tree did, via self-interaction with gravity).

**Human detects** what's intrinsically there.

**Same with quantum**:

State self-examines → $|c_i|^2$ exists

Observer measures → detects this property

Observer doesn't create probability (state did, via self-examination).

\subsection{Why Collapse Appears}

**Before external measurement**:

System in superposition (closed cycle through all states).

Self-examination: $|\psi\rangle\langle\psi|$ exists (intrinsic probability).

**But**: All branches present (closed structure).

**External measurement**:

**Opens** the closed structure (observer couples to system).

Selects one branch (according to probability $|c_i|^2$ that already existed).

**Collapse** = opening of pre-existing closed structure.

**Which branch?** Determined by intrinsic $|c_i|^2$ (from self-examination).

\textbf{Observer doesn't create probability} (system did).

\textbf{Observer opens structure} (according to pre-existing probability).

\section{The Complete Picture}

\subsection{Three Levels}

\textbf{Level 1 - Amplitude} (hidden):
- $c_i \in \mathbb{C}$ (complex)
- Phase-dependent (gauge choice)
- **Not intrinsic** (changes under gauge transformation)
- Not directly observable

\textbf{Level 2 - Self-Examination} (Δ=1):
- $|\psi\rangle$ examines self → $|\psi\rangle\langle\psi|$
- Creates $|c_i|^2$ (probability)
- **Intrinsic** (from self-pairing)
- **Gauge-invariant** (phase cancels)
- **Observable** (because invariant)

\textbf{Level 3 - External Detection}:
- Observer couples to system
- **Detects** pre-existing $|c_i|^2$ (not creates)
- Opens closed structure (selects branch)
- **Outcome distributed by** intrinsic probability

\subsection{Why This Resolves Paradoxes}

**Paradox**: Does probability exist before observation? (Realism vs. anti-realism)

**Resolution**:

Probability **exists intrinsically** (from self-examination, Level 2)

But: **Not as definite outcome** (all branches in closed cycle)

External observation: **Detects intrinsic probability** + **opens to definite outcome**

**Both**: Probability is real (intrinsic from self-examination)

**And**: Outcome is observer-dependent (which branch opens)

**No contradiction**: Property exists (intrinsic), outcome emerges (extrinsic).

\section{Why Squaring Specifically}

\subsection{Geometric Interpretation}

$|\psi|^2 = \psi^* \psi$ is **inner product** (state with itself):
\[
\langle\psi|\psi\rangle = \sum_i |c_i|^2
\]

This is: **Length squared** (in Hilbert space geometry).

**Probability** = normalized length squared (intrinsic geometric property).

\subsection{Information-Theoretic}

$|c_i|^2$ measures: **Information content** of component $|i\rangle$ in $|\psi\rangle$.

Shannon entropy: $S = -\sum_i |c_i|^2 \log|c_i|^2$

Uses $|c_i|^2$ (Born rule probabilities).

**Why squared?**

Because: Information is additive for independent systems.

$S(A \otimes B) = S(A) + S(B)$ requires $P = |\psi|^2$ (tensor product rule).

**Linear** ($P = |c_i|$) wouldn't give additivity.

**Cubic** ($P = |c_i|^3$) wouldn't normalize correctly.

**Only quadratic** satisfies:
\begin{itemize}
\item Unitarity (conservation)
\item Additivity (for composite systems)
\item Gauge invariance (phase cancellation)
\end{itemize}

\section{Self-Examination vs. External Observation}

\subsection{The Distinction}

\begin{center}
\begin{tabular}{lll}
\textbf{Aspect} & \textbf{Self-Examination} & \textbf{External Observation} \\ \hline
Actor & System itself & External observer \\
Operation & $|\psi\rangle$ with $|\psi\rangle$ & Observer with $|\psi\rangle$ \\
Result & Intrinsic property ($|c_i|^2$) & Definite outcome ($|i\rangle$) \\
When & Always (continuous) & Discrete events \\
Effect on state & None (property emerges) & Collapse (opens structure) \\
Determinism & Creates probability (Δ=1) & Samples probability (stochastic) \\
\end{tabular}
\end{center}

\textbf{Key}: Both are examination, different levels.

\textbf{Internal} (Δ=1): Creates property

\textbf{External} (measurement): Detects property + opens to outcome

\subsection{Why Both Needed}

**Self-examination alone**: Creates probability (all branches present, closed)

**External observation alone**: Undefined (what probability to use?)

**Together**:
\begin{enumerate}
\item System self-examines → probability exists intrinsically
\item Observer measures → detects probability, opens to outcome
\item Outcome distributed by intrinsic probability (Born rule)
\end{enumerate}

**Complete description** requires both levels.

\section{The Closure Argument}

\subsection{Why Δ=1 Creates Observable}

\begin{theorem}[Closure Creates Objectivity]
Properties created via one self-examination iteration (Δ=1) are:
\begin{enumerate}
\item \textbf{Gauge-invariant}: Phase/angle-independent
\item \textbf{Observer-independent}: All observers agree on value
\item \textbf{Objective}: Intrinsic to system, not observer-dependent
\end{enumerate}

\textbf{Therefore}: Externally observable (despite being internally generated).
\end{theorem}

\begin{proof}
**Self-pairing** ($\psi^* \psi$) creates gauge-invariant combination:

Under $|\psi\rangle \to U|\psi\rangle$ (unitary):
\[
|\psi\rangle\langle\psi| \to U|\psi\rangle\langle\psi|U^\dagger
\]

Expectation: $\langle i|U|\psi\rangle\langle\psi|U^\dagger|i\rangle$

For phase rotation $U = e^{i\alpha}$:
\[
e^{i\alpha}\langle i|\psi\rangle\langle\psi|i\rangle e^{-i\alpha} = \langle i|\psi\rangle\langle\psi|i\rangle
\]

**Cancellation** (phases eliminate).

**Result**: $|c_i|^2$ independent of gauge choice.

**All observers** (related by gauge transformations) **see same value**.

**Property is objective** (even though created internally).

**This objectivity enables external detection.**
\end{proof}

\subsection{Generalization}

\textbf{Any property} created via Δ=1 self-reference becomes objective/observable:

**Examples**:
\begin{itemize}
\item Mass: Particle self-interaction (via Higgs) → $m^2$ (intrinsic, observable)
\item Proper time: Spacetime point with self → $ds^2$ (intrinsic interval, observable)
\item Probability: State with self → $|\psi|^2$ (intrinsic, observable)
\item Energy: Field with self → $E^2 \sim \phi^2$ (in scalar field, intrinsic, observable)
\end{itemize}

**Pattern**: Self-reference once (Δ=1) → quadratic property → gauge-invariant → observable.

**This is universal** (Closure Principle at all levels).

\section{Why Observers See It}

\subsection{The Resolution}

**Original puzzle**: If probability is intrinsic (from self-examination), why do **we** (external observers) see it?

**Answer**:

**Because it achieved closure** (gauge-invariance through self-pairing).

Gauge-invariant properties are **the same for all observers** (objective).

**Therefore**: When we measure, we detect **what's objectively there** (intrinsic property that's observer-independent).

**Not**: Our observation creates probability

**But**: Our observation **couples to** pre-existing probability (that system created via self-examination).

\subsection{The Mechanism of Detection}

External measurement apparatus **couples** to system:

\[
|\psi\rangle_{\text{system}} \otimes |R\rangle_{\text{apparatus}} \to \sum_i c_i |i\rangle_S \otimes |i\rangle_A
\]

Apparatus becomes **correlated** with system (entangled).

**Which correlation?** Determined by system's intrinsic $|c_i|^2$.

**Apparatus detects** what's there (doesn't create it).

\textbf{Like}: Thermometer detects temperature (doesn't create it).

Temperature = intrinsic (from molecular self-interactions).

Thermometer couples → reads intrinsic property.

**Same with Born rule.**

\section{Implications}

\subsection{Quantum Mechanics is Self-Referential}

**Fundamentally**:

Probabilities emerge from **systems examining themselves** (not external observers).

**Measurement** = detecting these intrinsic properties + opening to outcome.

**Observer** plays role (opens structure) but **doesn't create** probability (system did, via self-examination).

\subsection{Realism Restored}

**Born rule exists** independently of observation (intrinsic from self-examination).

**But**: Outcomes are observer-dependent (which branch opens).

**Satisfies**:
\begin{itemize}
\item Realism (probability is real, objective)
\item Relationalism (outcomes depend on measurement context)
\end{itemize}

**Both** without contradiction.

\subsection{Connection to Dependent Origination}

**Mutual dependence** (pratītyasamutpāda):

State and observation **co-arise** (neither prior):
\begin{itemize}
\item State self-examines → probability exists
\item Observer measures → outcome emerges
\item Neither complete without other (reciprocal)
\end{itemize}

**Like Vijñāna ↔ Nāmarūpa**:
- Consciousness (observer) needs form (state)
- Form (state) needs consciousness (to be known)
- **Self-examination** (state with self) is what creates observable property
- External examination (observer with state) detects it

**Both examinations necessary** (internal creates property, external detects it).

\section{Conclusion}

\begin{center}
\fbox{\parbox{0.9\textwidth}{
\textbf{Born Rule Derived from Self-Examination}

$P(i) = |c_i|^2$ because:

1. Quantum states self-examine: $|\psi\rangle\langle\psi|$ (intrinsic operation)

2. One self-reference (Δ=1) creates stable property

3. Self-pairing cancels phase → gauge-invariant

4. Gauge-invariant = objective = externally observable

5. External observers detect (not create) this intrinsic property

6. Squaring is necessary (one self-reference) and sufficient (Δ=1 closes)

\textbf{Probability emerges from system's self-examination, not from external measurement.}

Intrinsic (from self-closure) yet observable (from gauge-invariance).

Born rule is Closure Principle applied to quantum states.

Quadratic structure universal (all Δ=1 properties).
}}
\end{center}

\vspace{1cm}

\textbf{This resolves}: Measurement problem (probability pre-exists, measurement detects it)

\textbf{This explains}: Why $|\psi|^2$ not $|\psi|$ or $|\psi|^3$ (Δ=1 structure)

\textbf{This unifies}: Born rule with all other quadratic structures in mathematics/physics

\textbf{Born rule is not postulate but theorem} (from Closure Principle + self-examination).

\end{document}
